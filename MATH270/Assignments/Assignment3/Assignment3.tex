\documentclass[11pt, letterpaper, twoside]{article}
\usepackage[letterpaper, portrait, left=1in, right=1in, top=1in, bottom=1in]{geometry}\usepackage{amsmath}
\usepackage{amssymb}
\usepackage{graphicx}
\usepackage{mathtools}% For bsmallmatrix
\usepackage[explicit]{titlesec}
\usepackage{epstopdf}
\usepackage{amsmath}
\usepackage{inputenc}
\usepackage{enumitem}
\usepackage{booktabs, multirow} %for borders and merged ranges
\usepackage{soul}% for underlines
\usepackage[table]{xcolor} % for cell colors
\newcommand\aug{\fboxsep=-\fboxrule\!\!\!\fbox{\strut}\!\!\!} %Use \aug to make a equal column for an augmented matrix. Ie. 1 & 2 & 3 & 4 & \aug & x \\
\begin{document}
\begin{titlepage}
	\centering
	\vspace*{60px}
	\hspace{0pt}
	\includegraphics[width=0.2\textwidth]{logo}\par\vspace{1cm}
	{\scshape\LARGE Athabasca University \par}
	\vspace{1cm}
	{\scshape\Large MATH 270\par}
	\vspace{1.5cm}
	{\huge\bfseries Assignment 3\par}
	\vspace{2cm}
	{\Large\itshape Stanley Zheng\par}
	\vfill
	{\large July 8, 2020\par}
	\vspace*{50px}
	\hspace{0pt}
\pagebreak
\end{titlepage}
\begin{enumerate}
\item \begin{enumerate}[label=(\alph*)] %Question 1
\item We will cofactor expand along row 1.
\begin{align*}
\det\begin{bmatrix}
3 & 3 & 1\\
1 & 0 & -4\\
1 & -3 & 5
\end{bmatrix}&=3\cdot\det\begin{bmatrix}
0 & -4\\
-3 & 5\\
\end{bmatrix}-3\cdot\det\begin{bmatrix}
1 & -4\\
1 & 5\\
\end{bmatrix}+1\det\begin{bmatrix}
1 & 0\\
1 & -3\\
\end{bmatrix}\\
&=3(\cdot5-(-4)\cdot(-3))-3(5\cdot1-(-4)\cdot1)+(1\cdot(-3)-0\cdot1)\\
&=3\cdot-12-9\cdot3-3\\
&=\boxed{-66}
\end{align*}
\item
Again, we will cofactor expand along row 1.
\begin{align*}\det\begin{bmatrix}
k+1 & k-1 & 7\\
2 & k-3 & 4\\
5 & k+1 & k
\end{bmatrix}&=(k+1)\cdot\det\begin{bmatrix}
k-3 & 4\\
k+1 & k\\
\end{bmatrix}-(k-1)\cdot\det\begin{bmatrix}
2 & 4\\
5 & k\\
\end{bmatrix}+7\cdot\det\begin{bmatrix}
2 & k-3\\
5 & k+1\\
\end{bmatrix}\\
&=(k+1)(k(k-3)-4(k+1))-(k-1)(2k-20)+7(2(k+1)-5(k-3))\\
&=(k^3-6k^2-11k-4)-(2k^2-22k+20)+(-21k+119)\\
&=\boxed{k^3-8k^2-10k+95}
\end{align*}
\end{enumerate}
\item
\begin{enumerate}[label=(\alph*)]
\item We can start by manipulating this matrix into row echelon form by dividing $R_1$ by 3, adding $2R_1$ to $R_3$, and swapping $R_2$ and $R_3$. Finally, we can divide $R_2$ by 5 and $R_3$ by -2.
\begin{align*}
\det A&=\begin{vmatrix}
3 & 6 & -9\\
0 & 0 & -2\\
-2 & 1 & 5
\end{vmatrix}\\
\frac{\det A}{3}&=\begin{vmatrix}
1 & 2 & -3\\
0 & 5 & -1\\
0 & 0 & -2
\end{vmatrix}\\
\frac{\det A}{30}&=\begin{vmatrix}
1 & 2 & -3\\
0 & 1 & -\frac{1}{5}\\
0 & 0 & 1
\end{vmatrix}
\end{align*}
Finally, we can find the determinant of our row echelon matrix by cofactor expanding on $R_3$.
\begin{align*}\frac{\det A}{30}&=1\det\begin{bmatrix}
1 & 2\\
0 & 1\\
\end{bmatrix}=1\\
\det A &= 30
\end{align*}
\item
We can simply manipulate the second matrix using elementary row operations until it is equal to the first, then observe the result. Let
$A=\begin{bsmallmatrix}
a & b & c\\
d & e & f\\
g & h & i
\end{bsmallmatrix}$. Then, we have $\det A = -6$. We can then set this equal to $A$, then manipulate $A$ into the matrix we are trying to find the determinant of. To do this, we first subtract $R_2$ from $R_1$, then multiply $R_2$ by -1.\pagebreak
\begin{align*}
\det A &= \begin{vmatrix}
a & b & c\\
d & e & f\\
g & h & i
\end{vmatrix}\\
\det A &=\begin{vmatrix}
a+d & b+e & c+f\\
d & e & f\\
g & h & i
\end{vmatrix}\\
-\det A&=
\begin{vmatrix}
a+d & b+e & c+f\\
-d & -e & -f\\
g & h & i
\end{vmatrix}
\end{align*}
Therefore, we have $\begin{vsmallmatrix}
a+d & b+c & c+f\\
-d & -e & -f\\
g & h & i
\end{vsmallmatrix}=\boxed{6}$
\item We can use elementary column operations on the first matrix to make it equal to the second matrix since elementary row operations do not change the discriminant of a matrix. First, we can divide $C_2$ by $t\cdot C_1$ and $C_3$ by $s\cdot C_1$.
$$\begin{vmatrix}a_1 & b_1 & c_1\\
a_2 & b_2 & c_2\\
a_3 & b_3 & c_3\end{vmatrix}=\begin{vmatrix}
a_1 & b_1 & c_1+rb_1\\
a_2 & b_2 & c_2+rb_2\\
a_3 & b_3 & c_3+rb_3
\end{vmatrix}$$
Next, we divide $C_3$ by $r\cdot C_2$.
$$\begin{vmatrix}a_1 & b_1 & c_1\\
a_2 & b_2 & c_2\\
a_3 & b_3 & c_3\end{vmatrix}=\begin{vmatrix}
a_1 & b_1 & c_1\\
a_2 & b_2 & c_2\\
a_3 & b_3 & c_3
\end{vmatrix}$$
\end{enumerate}
\item \begin{enumerate}[label=(\alph*)]
\item %Q3A
We can begin by finding the determinant of $A$ using cofactor expansion.
\begin{align*}
\det A &=\begin{vmatrix}
1 & 2 & 0\\
k & 1 & k\\
0 & 2 & 1
\end{vmatrix}\\
&=1\begin{vmatrix}
1 & k\\
2 & 1
\end{vmatrix}-2\begin{vmatrix}
k & k\\
0 & 1
\end{vmatrix}\\
&=1-2k-2k\\
&=1-4k
\end{align*}
Given that a matrix is invertible when its determinant is nonzero and real, we know that the matrix is invertible when $\boxed{k\neq\frac{1}{4}}$
\item \begin{enumerate}[label=\roman*.]
\item Recall that $\det(AB)=\det A \cdot \det B$. \\ We have $\det(-I\cdot A)$, and we know that $\det I=1$, so $\det(-1)=-\det(A)=\boxed{2}$
\item We know that $\det(AB)=\det A \cdot \det B$ and $A\cdot A^{-1}=I$. \\ Rearranging this, we get $\det(A\cdot A^{-1})=\det(I)=1$. \\ Finally, we have $\det(A^{-1})=\frac{1}{\det (A)}$, so $\det(A^{-1})=\boxed{-\frac{1}{2}}$
\item If we cofactor expand $A$ using the first row before $A$ is transposed, it is equivalent to cofactor expanding along the first column after $A$ is transposed. Therefore, we know that $\det A^T=\det A$. \\
We can use the property $\det(AB)=\det A \cdot \det B$ to break up the two parts. $\det 2I \cdot \det A=2\det A = \boxed{-4}$
\item $\det(A^3)$ Similarly to the previous questions, we can use the property \\$\det(AB)=\det A \cdot \det B$. We have $\det(A^3)=\det A \cdot \det A \cdot \det A=(-2)^3=\boxed{-8}$
\end{enumerate}
\end{enumerate}
\item \begin{enumerate}[label=(\alph*)]
\item %Q4A
We can find if the matrix is invertible by finding its discriminant. We can use cofactor expansion along $R_1$
$$\det A=2\begin{vmatrix}
1 & 0\\
3 & 6
\end{vmatrix}=12\\$$
The matrix is invertible since its determinant is nonzero. We then need to find the adjugate matrix of $A$ by cofactoring the matrix, then we can multiply by the discriminant to find the inverse.
\begin{align*}
A^{-1}&=\frac{1}{12}adj \begin{bmatrix}
2 & 0 & 0\\
8 & 1 & 0\\
-5 & 3 & 6
\end{bmatrix}\\
&=\frac{1}{12}\begin{bmatrix}
\begin{vmatrix}
1 & 0\\
3 & 6
\end{vmatrix}&
-\begin{vmatrix}
8 & 0\\
-5 & 6
\end{vmatrix}&
\begin{vmatrix}
8 & 1\\
-5 & 3
\end{vmatrix}\\
-\begin{vmatrix}
0 & 0\\
3 & 6
\end{vmatrix}&\begin{vmatrix}
2 & 0\\
-5 & 6
\end{vmatrix}&
-\begin{vmatrix}
2 & 0\\
-5 & 3
\end{vmatrix}\\
\begin{vmatrix}
0 & 0\\
1 & 0
\end{vmatrix}&-\begin{vmatrix}
2 & 0\\
8 & 0
\end{vmatrix}&\begin{vmatrix}
2 & 0\\
8 & 1
\end{vmatrix}
\end{bmatrix}^T\\
&=\frac{1}{12}\begin{bmatrix}
6 & -48 & 29\\
0 & 12 & -6\\
0 & 0 & 2\\
\end{bmatrix}^T\\
&=\begin{bmatrix}
\frac{1}{2}&0&0\\
-4&1&0\\
\frac{29}{12}&-\frac{1}{2}&\frac{1}{6}
\end{bmatrix}
\end{align*}
\item %Q4B
We can first rewrite this system of equations as three matrices, one for each variable.
\begin{align*}
D&=\begin{vmatrix}
1 & -4 & 1\\
4 & -1 & 2\\
2 & 2 & -3
\end{vmatrix}=1\cdot(-1)-(-4)\cdot(-16)+1\cdot10=-55\\
D_{x}&=\begin{vmatrix}
6 & -4 & 1\\
-1 & -1 & 2\\
-20 & 2 & -3
\end{vmatrix}=6\cdot(-1)-(-4)\cdot(43)+1\cdot(-22)=144\\
D_y&=\begin{vmatrix}
1 & 6 & 1\\
4 & -1 & 2\\
2 & -20 & -3
\end{vmatrix}=1\cdot(43)-6\cdot(-16)+1\cdot(-78)=61\\
D_z&=\begin{vmatrix}
1 & -4 & 6\\
4 & -1 & -1\\
2 & 2 & -20
\end{vmatrix}=1\cdot(22)-(-4)\cdot(-78)+6\cdot(10)=-230
\end{align*}
Cramer's rule states that $x=\frac{D_x}{D}$, $y=\frac{D_y}{D}$, and $z=\frac{D_z}{D}$. Therefore, we have
\begin{align*}
x&=\frac{144}{-55}=-\frac{144}{55}\\
y&=\frac{61}{-55}=-\frac{61}{55}\\
z&=\frac{-230}{-55}=\frac{46}{11}
\end{align*}
\end{enumerate}
\item \begin{enumerate}[label=(\alph*)]
\item %Q5A
Absolutely no clue.


\end{enumerate}
\end{enumerate}
\end{document}
