\documentclass[11pt, letterpaper, twoside]{article}
\usepackage[letterpaper, portrait, left=1in, right=1in, top=1in, bottom=1in]{geometry}\usepackage{amsmath}
\usepackage{amssymb}
\usepackage{graphicx}
\usepackage[explicit]{titlesec}
\usepackage{epstopdf}
\usepackage{amsmath}
\usepackage{inputenc}
\usepackage{enumitem}
\usepackage{booktabs, multirow} %for borders and merged ranges
\usepackage{soul}% for underlines
\usepackage[table]{xcolor} % for cell colors
\title{MATH 270 Assignment 1}
\date{Athabasca University\\May 6, 2020}
\author{Stanley Zheng\and ID\# 3486740}
\begin{document}
\begin{titlepage}
	\centering
	\vspace*{60px}
	\hspace{0pt}
	\includegraphics[width=0.2\textwidth]{logo}\par\vspace{1cm}
	{\scshape\LARGE Athabasca University \par}
	\vspace{1cm}
	{\scshape\Large MATH 270\par}
	\vspace{1.5cm}
	{\huge\bfseries Assignment 1\par}
	\vspace{2cm}
	{\Large\itshape Stanley Zheng\par}
	\vfill
	{\large May 14, 2020\par}
	\vspace*{50px}
	\hspace{0pt}
\pagebreak
\end{titlepage}
\begin{enumerate}
\item \begin{enumerate}[label=(\alph*)]
\item When we multiply the second equation by 2, we get the system
\begin{align*}
6x_1 + 2x_2 &=  -8 \\
6x_1 + 2x_2 &=  -8
\end{align*}

Equation 1 is equal to equation 2, and as a result, the two equations in this linear system are collinear and have an infinite amount of solutions.

To find the solutions, we can let $x_1=t$. Then,
\begin{align*}
6t-2x_2&=-8\\
x^2&=-3t-4
\end{align*}

Therefore, the solution set is $x_1=t$, $x_2=-3t-4$

\item We will attempt to make all of the equations collinear to equation 2. In order to make equation 1 equal equation 2, we can multiply equation 1 by 2. In order to make equation 3 collinear to equation 2, we can multiply equation 3 by $-\frac{3}{2}$. When these operations are performed, we get the system
\begin{align*}
6x-3y+6z&=-12 \\
6x-3y+6z&=-12 \\
6x-3y+6z&=-12 \\
\end{align*}

Since all the equations are the same, the solution set is infinite. We can find the solutions by letting $y=s$, and $z=t$. Then,
$$ 6x-3s+6t=-12$$
$$ x=\frac{1}{2}s-t-2$$

Therefore, we have the solutions $x=\frac{1}{2}s-t-2, y=t, z=s$
\end{enumerate}


\item We can form an augmented matrix from the system of linear equations to use Gaussian and Gaussian-Jordan elimination.

$$\begin{bmatrix}
0 & -2 & 3 & 1\\
3 & 6 & -3 & -2\\
6 & 6 & 3 & 5
\end{bmatrix}$$

We can eliminate rows to form the matrix into row echelon form. Our first step is to get a leading 1, which can be done by dividing $R_2$ by 3.

$$\begin{bmatrix}
0 & -2 & 3 & 1\\
1 & 2 & -1 & -\frac {2}{3}\\
6 & 6 & 3 & 5
\end{bmatrix}$$

Then, we can eliminate the 6 in $R_3$ by multiplying $R_2$ by $-6$ and adding. In addition, we can swap $R_1$ and $R_2$ to have a leading 1.

$$\begin{bmatrix}
1 & 2 & -1 &-\frac {2}{3}\\
0 & -2 & 3 & 1\\
0 & -6 & 9 & 9
\end{bmatrix}$$

To get a leading 1 for $R_2$, we can divide by -2.

$$\begin{bmatrix}
1 & 2 & -1 &-\frac {2}{3}\\
0 & 1 & -\frac {3}{2} & -\frac {1}{2}\\
0 & -6 & 9 & 9
\end{bmatrix}$$

Finally, we can multiply $R_2$ by 6 and add to $R_3$ to eliminate the -6, and our matrix is in row echelon form.

$$\begin{bmatrix}
1 & 2 & -1 &-\frac {2}{3}\\
0 & 1 & -\frac {3}{2} & -\frac {1}{2}\\
0 & 0 & 0 & 1
\end{bmatrix}$$

Since the matrix is in row echelon form, we can use Gaussian elimination and back-substitute. The corresponding linear equation is
\begin{align*}
a+2&b-c=\frac{2}{3}\\
&b-\frac{3}{2}=-\frac{1}{2}\\
0a+0&b+0c=1
\end{align*}

From the equation $0a+0b+0c=1$, it is evident that this system is inconsistent, and therefore has no solutions. Another way we could find this solution is through Gauss-Jordan elimination. We can find this by further eliminating our row echelon form matrix. Our first step is to multiply $R_2$ by -2 and add it to $R_1$.

$$\begin{bmatrix}
1 & 0 & 2 &\frac {1}{3}\\
0 & 1 & -\frac {3}{2} & -\frac {1}{2}\\
0 & 0 & 0 & 1
\end{bmatrix}$$

Our final step to put this equation into reduced row echelon form is to multiply $R_3$ by $-\frac{1}{3}$ and add to $R_1$ to eliminate the $\frac{1}{3}$, and to multiply $R_3$ by $\frac{1}{2}$ and add to $R_2$ in order to eliminate the $-\frac{1}{2}$.

$$\begin{bmatrix}
1 & 0 & 2 & 0\\
0 & 1 & -\frac {3}{2} &0\\
0 & 0 & 0 & 1

\end{bmatrix}$$
The corresponding system of linear equations is
$$a\hspace{6mm}+2c=0$$
$$\hspace{6mm}b-\frac {3}{2}c=0$$
$$0a+0b+0c=1$$

Again, it is evident that the system of equations is inconsistent because of the equation $0a+0b+0c=1$.

\item Since we are not allowed to introduce fractions, we can only start by adding and subtracting rows. We can start by subtracting $R_1$ by $R_3$ and replacing $R_3$.

$$\begin{bmatrix}
2 & 1 & 3\\
0 & -2 & -29\\
1 & 3 & 2
\end{bmatrix}$$

Next, we can multiply $R_3$ by $-2$ and add it to $R_1$ in order to eliminate the 2 and put the first row into reduced row echelon form. Since the first column is in reduced row echelon form, we can move the third row to the top.

$$\begin{bmatrix}
1 & 3 & 2\\
0 & -5 & -1\\
0 & -2 & -29\\
\end{bmatrix}$$

Then, we can multiply $R_3$ by -2, add it to $R_2$, and multiply the new $R_2$ by -1 in order to make a 1 in the second column.

$$\begin{bmatrix}
1 & 3 & 2\\
0 & 1 & -57\\
0 & -2 & -29\\
\end{bmatrix}$$

We can then eliminate $R_1$ by multiplying $R_2$ by $-3$ and adding to $R_1$, and we can eliminate $R_3$ by multiplying $R_2$ by 2 and adding to $R_3$.

$$\begin{bmatrix}
1 & 0 & 173\\
0 & 1 & -57\\
0 & 0 & 143\\
\end{bmatrix}$$

Finally, we can multiply $R_3$ by $\frac {57}{143}$ and add to $R_2$ in order to eliminate it, and we can multiply $R_3$ by $-\frac{173}{143}$ and add to $R_1$ in order to eliminate. In addition, we can divide $R_3$ by $143$ in order to make it zero. Our final answer for the reduced row echelon form is
$$\begin{bmatrix}
1 & 0 & 0\\
0 & 1 & 0\\
0 & 0 & 1\\
\end{bmatrix}$$
\item
\begin{enumerate}[label=(\alph*)]
\item We can start by splitting the addition into $2A^T$.

$A^T$=$\begin{bmatrix}
3 & -1 & 1\\
0 & 2 & 1
\end{bmatrix}$, so $2A^T$=$\begin{bmatrix}
6 & -2 & 2\\
0 & 4 & 2
\end{bmatrix}$

Then, $2A^T+C=\begin{bmatrix}
6 & -2 & 2\\
0 & 4 & 2
\end{bmatrix}+\begin{bmatrix}
1 & 4 & 2\\
3 & 1 & 5
\end{bmatrix}=\begin{bmatrix}
7 & 2 & 4\\
3 & 5 & 7
\end{bmatrix}$
\item Again, we can start by individually computing $D^T$ and $E^T$.

$D^T\begin{bmatrix}
1 & -1 & 3\\
5 & 0 & 2\\
2 & 1 & 4
\end{bmatrix}$ and $E^T$=$\begin{bmatrix}
6 & -1 & 4\\
1 & 1 & 1\\
3 & 2 & 3
\end{bmatrix}$

Then, $D^t-E^T=\begin{bmatrix}
1 & -1 & 3\\
5 & 0 & 2\\
2 & 1 & 4
\end{bmatrix}-
\begin{bmatrix}
6 & -1 & 4\\
1 & 1 & 1\\
3 & 2 & 3
\end{bmatrix}=\begin{bmatrix}
-5 & 0 & -1\\
4 & -1 & 1\\
-1 & -1 & 1
\end{bmatrix}$
\item We can begin by computing the subtraction inside of the parenthesis.

$D-E=\begin{bmatrix}
1 & 5 & 2\\
-1 & 0 & 1\\
3 & 2 & 4
\end{bmatrix}-\begin{bmatrix}
6 & 1 & 3\\
-1 & 1 & 2\\
4 & 1 & 3
\end{bmatrix}=\begin{bmatrix}
-5 & 4 & -1\\
0 & -1 & -1\\
-1 & 1 & 1
\end{bmatrix}$

Finally, transposing $(D-E)$, we get
$\begin{bmatrix}
-5 & 0 & -1\\
4 & -1 & 1\\
-1 & -1 & 1
\end{bmatrix}$

\item Start by calculating $B^T$ and $5C^T$.

We have $B^T=\begin{bmatrix}
4 & 0\\
-1 & 2\\
\end{bmatrix}$ and $5C^T=\begin{bmatrix}
5 & 15\\
20 & 1\\
10 & 25
\end{bmatrix}$

However, we cannot add $B^T$ and $5C^T$ since they have different dimensions; matrices must have identical dimensions in order to add or subtract. Therefore, $B^T+5C^T$ is undefined.

\item Again, we can start by splitting the subtraction into $\frac {1}{2}C^T$ and $\frac {1}{4}A$.

$\frac {1}{2}C^T=\begin{bmatrix}
\frac {1}{2} & \frac{3}{2}\\
2 & \frac{1}{2}\\
1 & \frac{5}{2}
\end{bmatrix}$ and $\frac{1}{4} A = \begin{bmatrix}
\frac{3}{4} & 0\\
-\frac{1}{4} & \frac{1}{2}\\
\frac{1}{4} & \frac{1}{4}
\end{bmatrix}$

$\frac{1}{2}C^T-\frac{1}{4}A=\begin{bmatrix}
\frac {1}{2} & \frac{3}{2}\\
2 & \frac{1}{2}\\
1 & \frac{5}{2}
\end{bmatrix}-
\begin{bmatrix}
\frac{3}{4} & 0\\
-\frac{1}{4} & \frac{1}{2}\\
\frac{1}{4} & \frac{1}{4}
\end{bmatrix}
=\begin{bmatrix}
-\frac{1}{4} & \frac{3}{2}\\
\frac{9}{4} & 0\\
\frac{3}{4} & \frac{9}{4}
\end{bmatrix}$

\item We found $B^T$ in a previous question, so we can subtract $B^T$ from $B$.

$B-B^T=\begin{bmatrix}
4 & -1\\
0 & 2\\
\end{bmatrix}-
\begin{bmatrix}
4 & 0\\
-1 & 2\\
\end{bmatrix}
=\begin{bmatrix}
0 & -1\\
1 & 0\\
\end{bmatrix}$

\item Similarly to previous questions, we can split the subtraction into $2E^T$ and $3D^T$

$2E^T=\begin{bmatrix}
12 & -2 & 8\\
2 & 2 & 2\\
6 & 4 & 6
\end{bmatrix}$, and $3D^T=\begin{bmatrix}
3 & -3 & 9\\
15 & 0 & 6\\
6 & 3 &12
\end{bmatrix}$

We have $2E^T-3D^T=\begin{bmatrix}
12 & -2 & 8\\
2 & 2 & 2\\
6 & 4 & 6
\end{bmatrix}-
\begin{bmatrix}
3 & -3 & 9\\
15 & 0 & 6\\
6 & 3 &12
\end{bmatrix}
=\begin{bmatrix}
9 & 1 & -1\\
-13 & 2 & -4\\
0 & 1 & -6
\end{bmatrix}$
\item Since we calculated $(2E^T-3D^T)$ in the previous question, we could transpose our previous answer.

$(2E^T-3D^T)^T=\begin{bmatrix}
9 & -13 & 0\\
1 & 2 & 1\\
-1 & -4 & -6
\end{bmatrix}$

\item Let's start with the parenthesis, $B\cdot A$.

$B\cdot A = \begin{bmatrix}
4 & -1\\
0 & 2\\
\end{bmatrix} \times
\begin{bmatrix}
3 & 0\\
-1 & 2\\
1 & 1
\end{bmatrix}$

However, this multiplication is not possible. In order to multiply two matrices $x\cdot y$, $x$ must have the same number of columns as the rows in $y$. In this case, $B$ has 2 columns, but $A$ has 3 rows. Therefore, $C(BA)$ is undefined.

\item We can start by finding $E^T$, then multiplying $D\cdot E^T$.

\hspace{6mm}$E^T=\begin{bmatrix}
6 & 1 & 4\\
1 & 1 & 1\\
3 & 2 & 3
\end{bmatrix}$

$D\cdot E^T = \begin{bmatrix}
1 & 5 & 2\\
-1 & 0 & 1\\
3 & 2 & 4
\end{bmatrix} \times
\begin{bmatrix}
6 & 1 & 4\\
1 & 1 & 1\\
3 & 2 & 3
\end{bmatrix}$

\hspace{12.5mm}$=\begin{bmatrix}
1\cdot 6 + 5\cdot 1+2\cdot 3 & &  1\cdot1+5\cdot1+2\cdot2 & & 1\cdot4+5\cdot3+2\cdot3\\
(-1)\cdot 6+0\cdot 1+1\cdot3 & & (-1)\cdot 1+0\cdot 1+2\cdot 2 & & (-1)\cdot 4+0\cdot 1+4\cdot 3\\
3\cdot6+2\cdot1+4\cdot 3 & & 3\cdot1+2\cdot1+4\cdot2 & & 3\cdot 4+2\cdot 1+4\cdot 3
\end{bmatrix}$


\hspace{12.5mm}$=\begin{bmatrix}
17 & 8 & 15\\
-3 & 3 & -1\\
32 & 7 & 26
\end{bmatrix}$

The trace of a matrix is the sum of the entries in the main diagonal, which in this case is $17+3+26=\boxed {46}$

\end{enumerate}
\item %Question 5
\begin{enumerate}[label=(\alph*)]

\item We can start by making the systems of linear equations homogeneous by moving all the variables to one side.
$$5x+y+z-K=0$$
$$x+7y+z-K=0$$
$$x+y+8z-K=0$$

We can then form an augmented matrix

$$\begin{bmatrix}
5 & 1 & 1 & -1 & 0\\
1 & 7 & 1 & -1 & 0\\
1 & 1 & 8 & -1 & 0
\end{bmatrix}$$

Next, we can start to manipulate the augmented matrix into reduced row echelon form. Multiply $R_3$ by -1 and add to $R_2$ to eliminate the 1, and multiply $R_3$ by -5 and add to $R_1$ to eliminate the 5.

$$\begin{bmatrix}
0 & -4 & -39 & 4 & 0\\
0 & 6 & -7 & 0 & 0\\
1 & 1 & 8 & -1 & 0
\end{bmatrix}$$

Since $R_3$ has a leading 1, we can swap it with $R_1$. We can also divide $R_2$ by 6 in order to make a leading 1.

$$\begin{bmatrix}
1 & 1 & 8 & -1 & 0\\
0 & 1 & -\frac{7}{6} & 0 & 0\\
0 & -4 & -39 & 4 & 0\\
\end{bmatrix}$$

We can use the leading 1 in $R_2$ to eliminate the other rows. We multiply $R_2$ by 4 and add to $R_3$ to eliminate the -4, and multiply $R_2$ by -1 and add to $R_1$ to eliminate the -1.

$$\begin{bmatrix}
1 & 0 & \frac{55}{6} & -1 & 0\\
0 & 1 & -\frac{7}{6} & 0 & 0\\
0 & 0 & -\frac{131}{3} & 4 & 0\\
\end{bmatrix}$$

We need to make another leading 1, so we can divide $R_3$ by $-\frac{131}{3}$

$$\begin{bmatrix}
1 & 0 & \frac{55}{6} & -1 & 0\\
0 & 1 & -\frac{7}{6} & 0 & 0\\
0 & 0 & 1 & -\frac{12}{131} & 0\\
\end{bmatrix}$$

We can multiply $R_3$ by $\frac{7}{6}$ and add to $R_2$ to eliminate the $-\frac{7}{6}$, and multiply $R_3$ by $-\frac{55}{6}$ to eliminate the $\frac{55}{6}$. After these operations, our matrix is in reduced row echelon form.

$$\begin{bmatrix}
1 & 0 & 0 & -\frac{21}{131} & 0\\
0 & 1 & 0 & -\frac{14}{131} & 0\\
0 & 0 & 1 & -\frac{12}{131} & 0\\
\end{bmatrix}$$

This corresponds to the homogeneous system of linear equations

\begin{align*}
x - \frac{21}{131}k &=  0 \\
y - \frac{14}{131}k &=  0\\
z-\frac{12}{131}k &= 0
\end{align*}

From this, it is evident that any one arbitrary parameter, whether it be $x$, $y$, $z$, or $k$, would provide enough information to calculate the other 3 variables.

\item Taking the systems of linear equations from the question above, we can solve for the smallest positive integer solution set

\begin{align*}
x - \frac{21}{131}k &=  0 \\
y - \frac{14}{131}k &=  0\\
z-\frac{12}{131}k &= 0
\end{align*}

If we let $k$ equal 131 to eliminate the fractions, we have $x=21, y=14, z=12$. We can attempt to find a lowest common divisor between these positive integers to see if a lower solution exists. 21 prime factors as $3\cdot7$, 14 has the prime factors $2\cdot 7$, and 12 has the prime factors $2\cdot 3^2$. Since there are no shared common factors, the solution set $\boxed{k=131, x=21, y=14, z=12}$ is the smallest positive integer solution set.

\item We can plug the solutions provided in the question into our system of linear equations and check if the equations are defined. The solution set is $k=262, x=42, y=28, z=24$.

\begin{align*}
42 - \frac{21}{131}\cdot262 &=  0 \\
28 - \frac{14}{131}\cdot262 &=  0\\
24-\frac{12}{131}\cdot262 &= 0
\end{align*}

Simplifying, we end up with
\begin{align*}
42 &= 42 \\
28&=  28\\
24 &= 24
\end{align*}

Seeing as the equations are all true and defined, the solution set provided in the question is included among the solutions of our system of equations.

\end{enumerate}
\end{enumerate}

\end{document}
