\documentclass[11pt, letterpaper, twoside]{article}
\usepackage[letterpaper, portrait, left=1in, right=1in, top=1in, bottom=1in]{geometry}\usepackage{amsmath}
\usepackage{amssymb}
\usepackage{graphicx}
\usepackage[explicit]{titlesec}
\usepackage{epstopdf}
\usepackage{amsmath}
\usepackage{inputenc}
\usepackage{enumitem}
\usepackage{booktabs, multirow} %for borders and merged ranges
\usepackage{soul}% for underlines
\usepackage[table]{xcolor} % for cell colors
\newcommand\aug{\fboxsep=-\fboxrule\!\!\!\fbox{\strut}\!\!\!} %Use \aug to make a equal column for an augmented matrix. Ie. 1 & 2 & 3 & 4 & \aug & x \\
\begin{document}
\begin{titlepage}
	\centering
	\vspace*{60px}
	\hspace{0pt}
	\includegraphics[width=0.2\textwidth]{logo}\par\vspace{1cm}
	{\scshape\LARGE Athabasca University \par}
	\vspace{1cm}
	{\scshape\Large MATH 270\par}
	\vspace{1.5cm}
	{\huge\bfseries Midterm\par}
	\vspace{2cm}
	{\Large\itshape Stanley Zheng\par}
	\vfill
	{\large August 16, 2020\par}
	\vspace*{50px}
	\hspace{0pt}
\pagebreak
\end{titlepage}
\noindent Solve the following system of 3 linear equations using Gaussian and Gauss-Jordan elimination.
\begin{align*}
x+2y+3z&=9\\
4x+5y+6z&=24\\
3x+y-2z&=4
\end{align*}

\vspace{0.5cm}

\noindent We can begin by manipulating this system into an augmented matrix.

\[\begin{bmatrix}
1 & 2 & 3 & \aug & 9\\
4 & 5 & 6 & \aug & 24\\
3 & 1 & -2 & \aug & 4
\end{bmatrix}\]

\noindent Next, we can begin manipulating this augmented matrix into row echelon form by adding \(-4R_1\) to \(R_2\) and adding \(-3R_1\) to \(R_3\). This creates a leading 1.

\[\begin{bmatrix}
1 & 2 & 3 & \aug & 9\\
0 & -3 & -6 & \aug & -12\\
0 & -5& -11 & \aug & -23
\end{bmatrix}\]

\noindent Finally, we let \(R_2=\frac{R_2}{-3}\) and \(R_3=5R_2+R_3\), and our matrix is in row echelon form.

\[\begin{bmatrix}
1 & 2 & 3 & \aug & 9\\
0 & 1 & 2 & \aug & 4\\
0 & 0& 1 & \aug & 3
\end{bmatrix}\]

\noindent From here, we can either solve for the leading variables (Gauss-Jordan elimination) or manipulate the matrix into reduced row echelon form (Gaussian elimination). We will begin with Gaussian elimination by adding \(-2R_2\) to \(R_1\)

\[\begin{bmatrix}
1 & 0 & -1 & \aug & 1\\
0 & 1 & 2 & \aug & 4\\
0 & 0 & 1 & \aug & 3
\end{bmatrix}\]

\noindent Finally, we let \(R_1=R_3+R_1\) and \(R_2=-2R_3+R_2\).

\[\begin{bmatrix}
1 & 0 & 0 & \aug & 4\\
0 & 1 & 0 & \aug & -2\\
0 & 0 & 1 & \aug & 3
\end{bmatrix}\]

\noindent Our solutions are \(\boxed{(x, y, z)=(4, -2, 3)}\). To solve using Gaussian-Jordan elimination, we can solve for the leading variables from our augmented matrix in row echelon form.

\[\begin{bmatrix}
1 & 2 & 3 & \aug & 9\\
0 & 1 & 2 & \aug & 4\\
0 & 0& 1 & \aug & 3
\end{bmatrix}\]

\begin{align*}
x+2y+3z&=9\\
y+2z&=4\\
z&=3
\end{align*}

\noindent Substituting \(z=3\), we get \(y+2(3)=4\), so \(y=-2\). Then, substituting \(y=-2\) and \(z=3\) into the first equation, we have \(x+2(-2)+3(3)=9\). Therefore, we have \(\boxed{(x, y, z)=(4, -2, 3)}\)

\end{document}