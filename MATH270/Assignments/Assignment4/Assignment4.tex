\documentclass[11pt, letterpaper, twoside]{article}
\usepackage[letterpaper, portrait, left=1in, right=1in, top=1in, bottom=1in]{geometry}\usepackage{amsmath}
\usepackage{amssymb}
\usepackage{graphicx}
\usepackage[explicit]{titlesec}
\usepackage{epstopdf}
\usepackage{amsmath}
\usepackage{inputenc}
\usepackage{enumitem}
\usepackage{booktabs, multirow} %for borders and merged ranges
\usepackage{soul}% for underlines
\usepackage[table]{xcolor} % for cell colors
\newcommand\aug{\fboxsep=-\fboxrule\!\!\!\fbox{\strut}\!\!\!} %Use \aug to make a equal column for an augmented matrix. Ie. 1 & 2 & 3 & 4 & \aug & x \\
\begin{document}
\begin{titlepage}
\centering
\vspace*{60px}
\hspace{0pt}
\includegraphics[width=0.2\textwidth]{logo}\par\vspace{1cm}
{\scshape\LARGE Athabasca University \par}
\vspace{1cm}
{\scshape\Large MATH 270\par}
\vspace{1.5cm}
{\huge\bfseries Assignment 4\par}
\vspace{2cm}
{\Large\itshape Stanley Zheng\par}
\vfill
{\large October 27, 2020\par}
\vspace*{50px}
\hspace{0pt}
\pagebreak
\end{titlepage}

\begin{enumerate}
\item 
\begin{enumerate}[label=\alph*)]
\item ‌‌\(\vec v + \vec w = (0+7, 1+4, -4+(-1), 1+(-2), 3) = (7, 5,-5,-1,5)\)
\item \(3(2\vec u - \vec v)=3(2(1,2,-3,5,0)-(0,4,-1,1,2))\)
\item \((3\vec u - \vec v)-(2\vec u + 4\vec w)\)
\item \(\frac{1}{2}((7,1,-4,-2,3) -5(0,4,-1,1,2) +2(1,2,-3,5,0))+(1,2,-3,5,0)\)
\end{enumerate}
\item \begin{enumerate}[label=\alph*]
\item
\begin{enumerate}[label=\roman*)]
\item The Euclidean distance can be found with the equation \(||u-v||=\sqrt{(u-v)\cdot(u-v)}\). 
Substituting our vectors, we have \(u-v=(1,2,-3,0)-(5,1,2,-2)=(-4,1,-1,2)\). The Euclidean distance is 
\[\sqrt{(-4,1,-1,2)(-4,1,-1,2)}=\sqrt{16+1-1+4}=\sqrt20\]

To find the angle, we must find \(||u||\) and \(||v||\) by square rooting and squaring.
\[||u|| = \sqrt{u^2} = \sqrt{1\cdot1+2\cdot2+(-3)\cdot(-3)+0\cdot0}=\sqrt{14}\]
\[||v||=\sqrt{v^2}=\sqrt{5\cdot5+1\cdot1+2\cdot2+(-2)\cdot(-2)}=\sqrt{34}\]

We know that \(\cos\theta=\frac{u\cdot v}{||u||\cdot||v||}.\)
\[\frac{(1,2,-3,0)(5,1,2,-2)}{\sqrt{14}\cdot\sqrt{34}}=\frac{5+2-3+0}{\sqrt{476}}=\frac{1}{\sqrt{476}}=\frac{1}{2\sqrt{119}}\]

Since \(\cos\theta>0\), we know that the angle between the vectors is acute.

\item We can use the same methodology as above. 
\[u-v=(0,1,1,1,2)-(2,1,0,-1,3)=(-2,0,1,2,-1)\]
Our Euclidean distance is 
\[\sqrt{(-2,0,1,2,-1)\cdot(-2,0,1,2,-1)}=\sqrt{(4+0+1+4+1)}=10\]

We know that \(\cos\theta=\frac{u\cdot v}{||u||\cdot||v||}.\) Substituting in, we have 
\begin{align*}
\cos\theta&=\frac{(0,1,1,1,2)\cdot(2,1,0,-1,3)}{\sqrt{(0,1,1,1,2)\cdot(0,1,1,1,2)}\cdot\sqrt{(2,1,0,-1,3)\cdot(2,1,0,-1,3)}}\\
&=\frac{0+1+0+-1+6}{\sqrt{0+1+1+1+4}+\sqrt{4+1+0+1+9}}\\
&=\frac{6}{\sqrt{105}}
\end{align*}

Again, since \(\cos\theta>0\), the angle \(\theta\) is acute.
\end{enumerate}
\item \begin{enumerate}[label=\roman*)]
\item We can find whether \(|u\cdot v|<||u||||v||\).
\[||u||=\sqrt{4\cdot4+1\cdot1+1\cdot1}=\sqrt{18}\]
\[||v||=\sqrt{1\cdot1+2\cdot2+3\cdot3}=\sqrt{14}\]
\[u\cdot v=4\cdot1+1\cdot2+1\cdot3=9\]

Then, \(|u\cdot v|=9<\sqrt{252}=||u||||v||\), so therefore, the Cauchy-Schwarz inequality holds.
\item Again, we can calculate \(||u||, ||v||\), and \(|u\cdot v|\).
\[||u||=\sqrt{1\cdot1+2\cdot2+1\cdot1+2\cdot2+3\cdot3}=\sqrt{19}\]
\[||v||=\sqrt{0\cdot0+1\cdot1+1\cdot1+5\cdot5+(-2)\cdot(-2)}=\sqrt{31}\]
\[u\cdot v=\sqrt{1\cdot0+2\cdot1+1\cdot1+2\cdot5+(-2)\cdot3}=\sqrt{7}\]

Then, \(|u\cdot v| = 7<\sqrt{539}=||u||||v||\), so therefore, the Cauchy-Schwarz inequality holds.
\end{enumerate}
\end{enumerate}
\item \begin{enumerate}[label=\alph*)]
\item We can begin by finding a point on either plane. 
For the plane \(2x-y+z=1\), we can let \(x, z=0\) so we find point \((0,-1,0)\).

Next, we need to find the distance between this point and the second plane, \(2x-y+z=-1\).
We can an equation to find the distance between this point and the plane.
\[\frac{|(2\cdot0+(-1)\cdot(-1)+1\cdot0+1)}{\sqrt{2^2+(-1)^2+1^2}}=\boxed{\frac{2}{\sqrt{6}}}\]
\item \begin{enumerate}[label=\roman*)]
Two vectors are orthogonal if \(|a\cdot b|=0\).
\[|a\cdot b|=-ab+ab=0\]
Therefore, these two vectors are orthogonal.
\item We can set the dot product of \(\vec v\) and a second vector, \(\vec{u}=(a,b)\) to 0.
\[|(a,b)\cdot(2,-3)|=0\]
\[2a-3b=0\]
A few possible values include \((a,b)=(3,2)\) or \((a,b)=(6,4)\)
\item We begin by finding orthogonal vectors to \((-3,4)\) like the previous question.
\[-3a+4b=0\]
One possible solution is \((a,b)=(4,3)\) or \((a,b)=(8,6)\).
To find the unit vector, we multiply a vector by the reciprocal of its magnitude.
\[\frac{1}{||(4,3)||}(4,3)=\frac{1}{\sqrt{4\cdot4+3\cdot3}}(4,3)=\frac{1}{5}(4,3)=\left(\frac{4}{5},\frac{3}{5}\right)\]
Since the original vector, (-3, 4) is of dimension 2, we can make our unit vector negative and it will still be orthogonal. 
As such, our two unit vectors are \(\boxed{\left(\frac{4}{5}, \frac{3}{5}\right), \left(-\frac{4}{5}, -\frac{3}{5}\right)}\)
\end{enumerate}
\end{enumerate} 
\item 
\begin{enumerate}[label=\alph*)]
\item 
We can begin by manipulating this system into an augmented matrix.
\[\begin{bmatrix}
1&3&-4&\aug&0\\
1&2&3&\aug&0
\end{bmatrix}\]
\[\begin{bmatrix}
1&3&-4&\aug&0\\
0&1&-7&\aug&0
\end{bmatrix}\]
\[\begin{bmatrix}
1&0&17&\aug&0\\
0&1&-7&\aug&0
\end{bmatrix}\]
Therefore, our solutions are \(x_1=-17x_3\) and \(x_2=7x_3\). We can form a vector: \((x_1, x_2, x_3)=(-17x_3, 7x_3, a)\). 
Finally, we can check whether this vector is orthogonal to the coefficient vectors. Note that \(a=x_3\)
\[1(-17x_3)+3(7x_3)-4(a)=-17x_3-4a+21x_3=0\]
\[1(-17x_3)+2(7x_3)+3(a)=-17x_3+14x_3+3a=0\]
\item \begin{enumerate}[label=\roman*)]
\item We can simply let the vectors be the coefficients for a homogenous system of linear equations.
\[-3x_1+2x_2-1x_3=0\]
\[-2x_2-2x_3=0\]
\item Each equation is a plane, and the intersections of the equations are the solutions.
We have 2 planes, so their intersection will always be a line.
%\item %DO NOT FORGET QUESTION 4bIII
\end{enumerate}
\end{enumerate}
\item \begin{enumerate}[label=\alph*)]
\item First, we calculate the vectors by the initial and terminal points.
We are given points \(P(1,-1,2),Q(0,3,4),R(6,1,8)\).
\[\vec{PQ}=0-1;3-(-1);4-2=(-1,4,2)\]
\[\vec{PR}=6-1;1-(-1);8-2=(5,2,6)\]
\[P=\frac{1}{2}|\vec{PQ}\cdot\vec{PR}|\]

Next, we calculate the cross product of our vectors \(\vec{PQ}, \vec{PR}\)
\[\vec r = \vec {PQ} \times \vec {PR}\]
\[=(4\cdot6-2\cdot2, 2\cdot5-(-1\cdot6), -1\cdot2-4\cdot5)=(20, 16, -22)\]

Finally, we calculate the magnitude of the vector and divide by two.. 
\[\vec r = \frac{\sqrt{20^2+16^2+(-22)^2}}{2}=\frac{\sqrt{1140}}{2}=\boxed{\sqrt{285}}\]
\item We have \(\vec u =(-1,2,4), \vec v=(3,4,-2), \vec w=(-1,2,5)\). 
We know that 
\[u\cdot(v\times w)=\begin{bmatrix}
    u_1&u_2&u_3\\
    v_1&v_2&v_3\\
    w_1&w_2&w_3
\end{bmatrix}=\begin{bmatrix}
    -1&2&4\\
    3&4&-2\\
    -1&2&5
\end{bmatrix}\]

With cofactor expansion, we have \(-1(25+4)-2(15-2)+4(6+4)=-29-26+40=-15\)
\end{enumerate}
\end{enumerate}
\end{document}