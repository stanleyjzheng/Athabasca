\documentclass[11pt, letterpaper, twoside]{article}
\usepackage[letterpaper, portrait, left=1in, right=1in, top=1in, bottom=1in]{geometry}\usepackage{amsmath}
\usepackage{amssymb}
\usepackage{graphicx}
\usepackage[explicit]{titlesec}
\usepackage{epstopdf}
\usepackage{amsmath}
\usepackage{inputenc}
\usepackage{enumitem}
\usepackage{booktabs, multirow} %for borders and merged ranges
\usepackage{soul}% for underlines
\usepackage[table]{xcolor} % for cell colors
\usepackage{esvect}% for long vectors (\vv)
\newcommand\aug{\fboxsep=-\fboxrule\!\!\!\fbox{\strut}\!\!\!} %Use \aug to make a equal column for an augmented matrix. Ie. 1 & 2 & 3 & 4 & \aug & x \\
\begin{document}
\begin{titlepage}
\centering
\vspace*{60px}
\hspace{0pt}
\includegraphics[width=0.2\textwidth]{logo}\par\vspace{1cm}
{\scshape\LARGE Athabasca University \par}
\vspace{1cm}
{\scshape\Large MATH 270\par}
\vspace{1.5cm}
{\huge\bfseries Final\par}
\vspace{2cm}
{\Large\itshape Stanley Zheng\par}
\vfill
{\large November 3, 2020\par}
\vspace*{50px}
\hspace{0pt}
\pagebreak
\end{titlepage}

\begin{enumerate}
\item Let \(A=(a, b, c), B=(d, e, f), C=(g, h, i)\).
We can now find the individual magnitudes \(\vv{A\cdot B}, \vv {B\cdot C}, \vv {A\cdot C}\)
\[\vv{AB}=A-B=(a,b,c)-(d,e,f)=(a-d, b-e, c-f)\]
\[\vv{BC}=B-C=(d,e,f)-(g,h,i)=(d-g,e-h,f-i)\]
\[\vv{AC}=A-C=(a,b,c)-(g,h,i)=(a-g,b-h,c-i)\]

Then, we can find the magnitude 
\[||\vv{AB}|| = \sqrt{(a-d)^2+(b-e)^2+(c-f)^2}\]
\[||\vv{BC}|| = \sqrt{(d-g)^2+(e-h)^2+(f-i)^2}\]
\[||\vv{AC}|| = \sqrt{(a-g)^2+(b-h)^2+(c-i)^2}\]

Our inequality is \(||\vv{AC}||^2\leq ||\vv{AB}||^2+||\vv{BC}||^2\). Substituting in, we have 
\[(a-d)^2+(b-e)^2+(c-f)^2\leq(d-g)^2+(e-h)^2+(f-i)^2+(a-g)^2+(b-h)^2+(c-i)^2\]

This vector is similar to the pythagorean theorem/inequality
\item 
\[\begin{bmatrix}
    1&2&3\\
    4&5&6\\
    7&8&9
\end{bmatrix}\]

We can find the eigenvectors and eigenvalues of the matrix.

We know that for matrix \(A\), there is a \(\lambda\) for which \(\det (A-\lambda I)=0\).
We can find the determinant with cofactor expansion and solve for \(\lambda\)
\begin{align*}
0&=\det \begin{bmatrix}
    1-\lambda&2&3\\
    4&5-\lambda&6\\
    7&8&9-\lambda
\end{bmatrix}\\
=(1-\lambda)((5-\lambda)(9-\lambda)-&(8\cdot 6))-2(4(9-\lambda)-7\cdot6)+3(4\cdot8-7(5-\lambda))\\
&=-\lambda^3+15\lambda^2-11\lambda-3-8\lambda-12+21\lambda-9\\
&=-\lambda^3+15\lambda^2+18\lambda
&=-\lambda(\lambda^2-15\lambda-18)
\end{align*}
We have one root, \(\lambda=0\). 
From here, we can use the quadratic formula to find the other roots.
\[\lambda=\frac{15\pm\sqrt{(-15)^2-4(1)(-18)}}{2}=\frac{15\pm3\sqrt{33}}{2}\]

Therefore, our eigenvalues are \(\lambda=0, \frac{15\pm3\sqrt{33}}{2}\)

Next, to find our eigenvectors, we can use Gaussian Elimination. 
For each eigenvalue \(\lambda\), we have \((A-\lambda I)\vv{x}=\vv{0}\).
To find \(\vec x\), we can create an augmented matrix consisting of \(A-\lambda I | 0\).

First, we let \(\lambda=0\).
\[
\begin{bmatrix}
    1-0&2&3&\aug&0\\
    4&5-0&6&\aug&0\\
    7&8&9-0&\aug&0
\end{bmatrix}
\]
\[
\begin{bmatrix}
    1&2&3&\aug&0\\
    0&-3&-6&\aug&0\\
    0&-6&-12&\aug&0
\end{bmatrix}
\]
\[
\begin{bmatrix}
1&0&-1&0\\
0&1&2&0\\
0&0&0&0
\end{bmatrix}
\]
This gives us the following linear system.
\[x_1-x_3=0\]
\[x_1+2x_2=0\]
Then, our eigenvector \(\vec x\) is given by 
\[\vv x=\begin{bmatrix}
    x_3\\
    -2x_3\\
    x_3\\
\end{bmatrix}=x_3\begin{bmatrix}
    1\\
    -2\\
    1
\end{bmatrix}\]

Next, we can let \(\lambda=\frac{15+3\sqrt{33}}{2}\)
\[
\begin{bmatrix}
    1-\frac{15+3\sqrt{33}}{2}&2&3&\aug&0\\
    4&5-\frac{15+3\sqrt{33}}{2}&6&\aug&0\\
    7&8&9-\frac{15+3\sqrt{33}}{2}&\aug&0
\end{bmatrix}
\]
\[
\begin{bmatrix}
    1&-\frac{4}{13+3\sqrt{33}}&-\frac{6}{13+3\sqrt{33}}&\aug&0\\
    4&5-\frac{15+3\sqrt{33}}{2}&6&\aug&0\\
    7&8&9-\frac{15+3\sqrt{33}}{2}&\aug&0
\end{bmatrix}
\]
\[
\begin{bmatrix}
    1&-\frac{4}{13+3\sqrt{33}}&-\frac{6}{13+3\sqrt{33}}&\aug&0\\
    0&\frac{-9\sqrt{33}-33}{8}&\frac{9\sqrt{33}+57}{16}&\aug&0\\
    0&\frac{21\sqrt{33}+165}{32}&-\frac{-33\sqrt{33}-177}{64}&\aug&0
\end{bmatrix}
\]
\[
\begin{bmatrix}
    1&-\frac{4}{13+3\sqrt{33}}&-\frac{6}{13+3\sqrt{33}}&\aug&0\\
    0&1&\frac{-3\sqrt{33}-11}{44}&\aug&0\\
    0&0&0&\aug&0
\end{bmatrix}
\]
\[
\begin{bmatrix}
    1&0&\frac{11-3\sqrt{33}}{22}&\aug&0\\
    0&1&\frac{-3\sqrt{33}-11}{44}&\aug&0\\
    0&0&0&\aug&0
\end{bmatrix}
\]
This gives us the following systems of equations.
\[x_1+\frac{11-3\sqrt{33}}{22}x_3=0\]
\[x_2+\frac{-3\sqrt{33}-11}{44}x_3=0\]
Thus, 
\[\vec{x}=x_3\begin{bmatrix}
    -\frac{11-3\sqrt{33}}{22}\\
    -\frac{-3\sqrt{33}-11}{44}\\
    1
\end{bmatrix}\]

Finally, we can let \(\lambda=\frac{15-3\sqrt{33}}{2}\)
\[
\begin{bmatrix}
    1-\frac{15-3\sqrt{33}}{2}&2&3&\aug&0\\
    4&5-\frac{15-3\sqrt{33}}{2}&6&\aug&0\\
    7&8&9-\frac{15-3\sqrt{33}}{2}&\aug&0
\end{bmatrix}
\]
\[
\begin{bmatrix}
    1&\frac{3\sqrt{33}+13}{32}&\frac{9\sqrt{33}+39}{64}&\aug&0\\
    4&5-\frac{15-3\sqrt{33}}{2}&6&\aug&0\\
    7&8&9-\frac{15-3\sqrt{33}}{2}&\aug&0
\end{bmatrix}
\]
\end{enumerate}
\end{document}