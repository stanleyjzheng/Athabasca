\documentclass[11pt, letterpaper, twoside]{article}
\usepackage[letterpaper, portrait, left=1in, right=1in, top=1in, bottom=1in]{geometry}\usepackage{amsmath}
\usepackage{amssymb}
\usepackage{graphicx}
\usepackage{mathtools}% For bsmallmatrix
\usepackage[explicit]{titlesec}
\usepackage{epstopdf}
\usepackage{amsmath}
\usepackage{inputenc}
\usepackage{enumitem}
\usepackage{booktabs, multirow} %for borders and merged ranges
\usepackage{soul}% for underlines
\usepackage[table]{xcolor} % for cell colors
\newcommand\aug{\fboxsep=-\fboxrule\!\!\!\fbox{\strut}\!\!\!} %Use \aug to make a equal column for an augmented matrix. Ie. 1 & 2 & 3 & 4 & \aug & x \\
\begin{document}
\begin{titlepage}
	\centering
	\vspace*{60px}
	\hspace{0pt}
	\includegraphics[width=0.2\textwidth]{logo}\par\vspace{1cm}
	{\scshape\LARGE Athabasca University \par}
	\vspace{1cm}
	{\scshape\Large MATH 265\par}
	\vspace{1.5cm}
	{\huge\bfseries Midterm\par}
	\vspace{2cm}
	{\Large\itshape Stanley Zheng\par}
	\vfill
	{\large September 25, 2020\par}
	\vspace*{50px}
	\hspace{0pt}
\pagebreak
\end{titlepage}
\begin{enumerate}
\item We know that \(f\cdot g = f(g(x))\). Plugging in \(f(x)\) and \(g(x)\), we have
\[f\cdot g=\sqrt{\sqrt{x-10}+10}\]
The domain of \(f\cdot g\) is \([10, \infty)\)

\item We can begin by taking the derivative.
\[y=8x^{-2}\]
\[\frac{dy}{dx}=-\frac{16}{x^3}\]
Plugging in \(x=1\), we have a slope of \(-\frac{16}{1^3}=-16\). To find the equation, we can simply substitute our values \((x,y)=(1,8)\).

\[y=-16x+b\]
\[8=-16(1)+b\]
\[b=24\]

Therefore, our final equation for the tangent line at point \(P(1,8)\) is \(\boxed{y=-16x+24}\)

\item We can begin by substituting in gradually smaller values of \(x\) approaching 0. 
\[f(x)=\frac{x^2+x}{4\sqrt{x^3+x^2}}\]
\[f(0.1)=\frac{0.1^2+0.1}{4\sqrt{0.1^3+0.1^2}}\approx0.262\]
\[ f(0.01)=\frac{0.01^2+0.01}{4\sqrt{0.01^3+0.01^2}}\approx0.2512\]

We can see the values approaching \(\frac{1}{4}\).

Next, we can substitute values increasing approaching \(x=0\).

\[f(x)=\frac{x^2+x}{4\sqrt{x^3+x^2}}\]
\[f(-0.1)=\frac{-0.1^2+-0.1}{4\sqrt{(-0.1)^3+(-0.1)^2}}\approx-0.237\]
\[f(-0.01)=\frac{(-0.01)^2+(-0.01)}{4\sqrt{(-0.01)^3+(-0.01)^2}}\approx-0.248\]

We can see that the values are approaching \(-\frac{1}{4}\). 

Now, we have \(\lim_{x\to0^+}=\frac{1}{4}\) and \(\lim_{x\to0^-}=-\frac{1}{4}\). Therefore, this limit has no 2-sided definition.

\item We can begin by taking the derivative over \(x\) and isolating \(\frac{dy}{dx}\). Our first step is to apply the product rule to both sides.
\begin{align*}
\frac{d}{dx}(y\sin(3x))&=\frac{d}{dx}(x\cos(3y))\\
\frac{dy}{dx}\sin(3x)+y\frac{d}{dx}\sin(3x)&=\cos(3y)+x\frac{d}{dx}\cos(3y)\\
\frac{dy}{dx}\sin(3x)+3y\cos(3x)&=\cos(3y)-3x\frac{dy}{dx}\sin(3y)\\
\end{align*}

Next, we can solve for \(\frac{dy}{dx}\) by isolating and then common factoring.
\begin{align*}
\frac{dy}{dx}(\sin(3x)+3x\sin(3y))=\cos(3y)-3y\cos(3x)\\
\frac{dy}{dx}=\frac{\cos(3y)-3y\cos(3x)}{\sin(3x)+3x\sin(3y)}
\end{align*}

To find the slope of the line, we can substitute the values provided in the question, \((x.y)=(\frac{\pi}{3}, \frac{\pi}{6})\).
\begin{align*}
	\frac{dy}{dx}&=\frac{\cos\left(\frac{3\pi}{6}\right)-3\left(\frac{\pi}{6}\right)\cos\left(\frac{3\pi}{3}\right)}{\sin\left(\frac{3\pi}{3}\right)+3\left(\frac{\pi}{3}\right)\sin\left(\frac{3\pi}{6}\right)}\\
	&=\frac{0-\left(-\frac{\pi}{2}\right)}{0+\pi}\\
	&=\frac{1}{2}
\end{align*}

From this, the equation of our line is \(y=\frac{1}{2}x+b\). 
We can once again substitute in our values \((x.y)=(\frac{\pi}{3}, \frac{\pi}{6})\) in order to solve for \(b\), the \(y\)-intercept.

\[\frac{\pi}{6}=\frac{1}{2}\cdot\frac{\pi}{3}+b\]
\[b=0\]

Therefore, our final tangent line equation is \(\boxed{y=\frac{1}{2}x}\).

\newpage

\item \begin{enumerate}[label=\alph*)]
\item Rational functions are continuous everywhere except where we have division by zero. 
Factoring the denominator, we have \(\frac{x+2}{(x+2)(x-2)}\). 
Therefore, the function is discontinuous at \(x=\pm2\) since the denominator is zero and undefined.

Alternatively, we can use the definition of continuity. 
The definition states that a function \(f(x)\) is continuous at \(x=a\) if \(\lim_{x\to a}f(x)=f(a)\).
However, at \(x=\pm2\), \(f(x)\) is undefined. Therefore, \(\lim_{x\to a}f(x)\neq f(a)\), and as a result, \(f(x)\) is discontinuous at these two points.
\item Let's begin with the limit for \(x=2\). 
We can substitute in progressively smaller values of \(x\) approaching 2 into \(f(x)\).
\vspace{0.1cm}

\(f(2.1)=\frac{4.1}{0.41}=10\), \(f(2.01)=\frac{4.01}{0.0401}=100, f(2.001)=\frac{4.001}{0.004001}=1000\)\\

We can see that as \(x\) approaches 2 from the right side, the denominator approaches 0 from the positive, while the numerator approaches 4. 
As a result, \(\lim_{x\to2^+}\frac{x+2}{x^2-4}=\infty\).\\

Next, we can substitute values of \(x\) getting larger approaching 2.

\(f(1.9)=\frac{3.9}{-0.39}-10, f(1.99)=\frac{3.99}{0.0309}=-100, f(1.999)=\frac{3.999}{0.003009}=-1000\)\\

Again, we can see that as \(x\) approaches 2 from the left side, the denominator approaches 0 from the negative, while the numerator approaches 4.

Therefore, \(\lim_{x\to2^-}\frac{x+2}{x^2-4}=-\infty\)

Finally, we have \(x=-2\). We can begin by trying to factor and simply the fraction.

\[\lim_{x\to-2}\frac{x+2}{x^2-4}=\lim_{x\to-2}\frac{1}{x-2}\]

Then, simply plugging in \(x=-2\), we have \(\lim_{x\to-2}\frac{1}{x-2}=-\frac{1}{4}\). 
Since the limit can be found algebraically, the limit is defined from both sides.
\end{enumerate}
\item Let the distance from 2nd base to the runner be \(x\), the distance from first base to second base be \(y\), and the distance from home plate to the runner be \(z\).

We can find the distance \(z\) through the pythagorean theorem.
\[z^2=90^2+40^2\]
\[z=10\sqrt{97}\]

To find the speed of \(z\), we can differentiate over \(t\).
\begin{align*}
z^2&=90^2+x^2\\
\frac{dz}{dt}2z&=\frac{dx}{dt}2x\\
\frac{dz}{dt}&=\frac{dx}{dt}\cdot\frac{x}{z}
\end{align*}
We can then begin to substitute in, \(\frac{dx}{dt}=24ft/s, z=10\sqrt{97}ft, x=40ft\)
\[\frac{dz}{dt}=24\cdot\frac{40}{10\sqrt{97}}=\boxed{\frac{96\sqrt{97}}{97}}\]

\end{enumerate}
\end{document}
