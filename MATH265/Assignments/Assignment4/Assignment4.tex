\documentclass[11pt, letterpaper, twoside]{article}
\usepackage[letterpaper, portrait, left=1in, right=1in, top=1in, bottom=1in]{geometry}\usepackage{amsmath}
\usepackage{amssymb}
\usepackage{graphicx}
\usepackage[explicit]{titlesec}
\usepackage{epstopdf}
\usepackage{amsmath}
\usepackage{inputenc}
\usepackage{enumitem}
\usepackage{booktabs, multirow} %for borders and merged ranges
\usepackage{soul}% for underlines
\usepackage[table]{xcolor} % for cell colors
\newcommand\aug{\fboxsep=-\fboxrule\!\!\!\fbox{\strut}\!\!\!} %Use \aug to make a equal column for an augmented matrix. Ie. 1 & 2 & 3 & 4 & \aug & x \\
\begin{document}
\begin{titlepage}
\centering
\vspace*{60px}
\hspace{0pt}
\includegraphics[width=0.2\textwidth]{logo}\par\vspace{1cm}
{\scshape\LARGE Athabasca University \par}
\vspace{1cm}
{\scshape\Large MATH 265\par}
\vspace{1.5cm}
{\huge\bfseries Assignment 4\par}
\vspace{2cm}
{\Large\itshape Stanley Zheng\par}
\vfill
{\large September 3, 2020\par}
\vspace*{50px}
\hspace{0pt}
\pagebreak
\end{titlepage}
\begin{enumerate}
\item 
\begin{enumerate}[label=\alph*)]
\item ‌‌ 

\vspace{0.2cm}
\includegraphics[width=0.7\textwidth]{q1}\par

The domain of the function \(g\) is \((-\infty, w)\cup(w, x)\cup(x, y]\)
\item The function \(g\) is continuous from \((-\infty, w)\cup(w, x)\cup(x, y]\).
\item The local maximum ix at \(x=0\), since the derivative is 0.
\item The local minimum is at two points, \(x\) and \(z\), where the derivative is zero.
\item There is no absolute maximum as the derivative at \(y\) is undefined, and \(y\) is not included in the domain.
\item The absolute minimum is at \(v\). It appears as though the graph has negative slope as \(x\) decreases beyond \(v\) to infinity.
\end{enumerate}
\item %q2
The first condition specifies that this is an odd function, so we must verify this by reflecting over the origin.

\item %q3
Our first step is to find the domain of the function.
Setting the denominator to zero, we have \(x^2+6=0\), so our domain is \(x\neq\pm\sqrt6\)

Next, we can find the intercepts. We can find the \(x\) intercept by letting \(y=0\).
We have \(x^2=4\), so our \(x\)-intercepts are \(x=\pm2\).
Likewise, our \(y\)-intercepts are \(-\frac{4}{6}\).

We can then test if the function is even or odd. 
We can test whether it is even or not by simplifying \(f(-x)\). 
If \(f(-x)=f(x)\), then our function is even.
\[f(-x)=\frac{(-x)^2-4}{(-x)^2+6}\]
Since \(-x\) is squared, we know that \(f(-x)=f(x)\), and therefore, the function is even.

Our function is not periodic, but it does have a horizontal asymptote. 
At extremely large values of \(x\), our function will be approximately equal to \(\frac{x^2}{x^2}\), so we have a horizontal asymptote at \(x=1\).

Next, we find the derivative of \(f(x)\).

\[f^\prime(x)=\frac{2x(x^2+6)-2x(x^2-4)}{(x^2+6)^2}\]
\[f^\prime(x)=\frac{20x}{(x^2+6)^2}\]

Since \(f^\prime(x)<0\) when \(x<0\) and \(f^\prime(x)>0\) when \(x>0\), we know that the function is decreasing on \((-\infty, 0)\) and increasing on \(0, \infty\).

Our only critical number is \(x=0\), where \(f^\prime\) changes from positive to negative.
We know it is a local minimum since the derivative to the right is positive, but negative to the left.

We can find concavity with the second derivative of the function. 

\[f^{\prime\prime}(x)=-\frac{60x^2-120}{(x^2+6)^3}\]

From this, the curve is concave upward at \((-\sqrt2, \sqrt2)\) and points of concavity are \(\pm\sqrt2\).

Finally, we can draw the graph of the function.

%GRAPH

\item %q4 (no check)
First, we need to find relative distances for swimming and jogging.
Let's let \(\angle PWE\) be \(\theta\).
We can create a triangle by connecting points \(PWE\) with straight lines.
We know from Thale's theorem that an angle inscribed across the diameter of a circle is always \(90^\circ\), so this triangle is a right triangle.
Then, the swimming distance, line \(PW\), is \(3\cos\theta\).
The arc length for jogging is \(\theta r\), and our diameter is 3km, so the jogging distance is \(2\cdot1.5\theta\). 
We must multiply by 2 since the angle is not measured at the center of the circle, but rather subtends it.

We know that critical points are found where the derivative of a function is 0 or undefined.
To find the derivative, we can find an equation for the time of the journey, then differentiate.
\[\frac{1}{24}\cdot3\theta+\frac{1}{3}\cdot3\cos\theta\]
\[\frac{\theta}{8}+\cos\theta\]
We can then differentiate and find zeros.
\[0=\frac{1}{8}-\sin\theta\]
\[\sin\theta=\frac{1}{8}\]
\[\sin^{-1}\left(\frac{1}{8}\right)=\theta\]
We have \(\theta\approx0.125\). 
To find the distance to jog, we multiply by 3.
The final distance Peter should jog to arrive at point W in the least amount of time is \(\boxed{0.376\text{km}}\)

\item %q5
\begin{enumerate}[label=\alph*)]
\item Graph is as follows:

\includegraphics[width=0.6\textwidth]{q5}\par\vspace{1cm}
\item We can evaluate the function at various reference angles, since we know that \(x^2\) will always be positive and \(\sin x\) has a maximum value of 1.

When \(x=0\), we have \(\sin0-(0^2)=0\).

When \(x=\frac{\pi}{2}\), we have \(\sin\frac{\pi}{2}-(\frac{\pi}{2})^2\), which is positive.

When \(x=\frac{\pi}{3}\), we have \(\sin\frac{\pi}{3}-(\frac{\pi}{3})^2\), which is negative.

When \(x=\frac{\pi}{4}\), we have \(\sin\frac{\pi}{4}-(\frac{\pi}{4})^2\), which is positive.

Then, we know that the interval in which \(x\) is nonzero is \([\frac{\pi}{4}, \frac{\pi}{3}]\)

\item 
In order to use Newton's method, we must have the function in the form \(f(x)=0\), so our function is \(\sin x - x^2=0\).

We can now write down the general formula for Newton's method.

\[x_{n+1}=x_n-\frac{\sin x_n-x_n^2}{\cos x_n -2x_n}\]

Next, we can get our first approximation by substituting our lower bound, \(\frac{\pi}{3}\).

\[x_1=\frac{\pi}{3}-\frac{\sin(\frac{\pi}{3})-\left(\frac{\pi}{3}\right)^2}{\cos(\frac{\pi}{3})-2\left(\frac{\pi}{3}\right)}\approx0.902567586\]

We must do this again, but substituting \(x_n\) with 0.902567586.

\[x_2=0.902567586-\frac{\sin0.902567586-\left(0.902567586\right)^2}{\cos0.902567586-2\left(0.902567586\right)}\approx0.8775090289\]

Again, we can substitute \(x_n\) with 0.8775090289.

\[x_3=0.8775090289-\frac{\sin0.8775090289-\left(0.8775090289\right)^2}{\cos0.8775090289-2\left(0.8775090289\right)}\approx0.8767269756\]

After one more repetition, we get \(x_4=0.876726215396\), which has 6 digits consistent with \(x_3\).
Therefore, \(x\) has a nonzero solution of \(\boxed{x=0.876726}\)

\end{enumerate}

\item %q6
We can begin by finding the integral of the first function to see if the ant walks 4cm in the first second.

\[\int_0^1(5t)dt=\frac{5t^2}{2}=\frac{5}{2}\]

The ant can only walk 2.5cm in the first second, therefore, we need to find the integral of the second function.

\[\int_1^t(6\sqrt t-\frac{1}{t})dt=4t^{\frac{3}{2}}-\ln t\]

We can then add our first integral to the second integral and equate it to 4cm.

\[4=\frac{5}{2}+4t^{\frac{3}{2}}-\ln t \]

We have \(t\approx0.266\) for our second function. 
Adding 1 second from our first function, the ant travelled for \(\boxed{1.266s}\).

\item %q7
\item %q8
%def taken from http://www.sosmath.com/calculus/diff/der11/der11.html
Mean value theorem states that \(f(x)\) is defined and continuous on the interval \([a, b]\) and is differentiable on interval \((a,b)\),
then there is at least one number \(c\) such that \(f^\prime(c)=\frac{f(b)-f(a)}{b-a}\).

For this question, we can let \(f(x)=\cos(x)\). 
Plugging into our equation, we have 
\[f^\prime(c)=\frac{\cos(a)-\cos(b)}{b-a}\]

We can put absolute value to the left-hand side as well as the denominator and numerator.
\[|f^\prime(c)|-\frac{|\cos(a)-\cos(b)|}{|b-a|}\]

We know that \(f(c)=\cos(c)\), so \(f^\prime(c)=-\sin(c)\), and the domain is \([-1, 1]\).
Therefore, we have 

\[|a-b|\geq |cos\prime(c)||a-b|\]

This is since the largest possible value of of \(\cos^\prime(c)\) and \(|a-b|\) is 1.

\item %q9 
\begin{enumerate}[label=\alph*]
\item
We can use the intermediate value theorem to find possible intervals of a root. 
To this, we can evaluate the function at a few points. 
Looking at the equation, it is not defined when \(x<0\). 
We can start with points \(x=0, 1, 2\).

When \(x=0\): \(f(0)=3\cdot0^3+\sqrt{0}-2=-2\)

When \(x=0.25\): \(f(0.25)=3\cdot0^3+\sqrt{0.25}-2=-1.3125\)

When \(x=1\): \(f(1)=3\cdot1^3+\sqrt(1)-2=2\)

\item
We can use Rolle's theorem and argue by contradiction.
It states that for a function with two real roots, \(a\) and \(b\), we would have \(f(a)=f(b)=0\).

Since \(f\) is a polynomial, it must be differentiable on \((a, b)\) and continuous on \([a,b]\).

Then, we need to find a number \(c\) between \(a\) and \(b\) such that \(f^\prime(c)=0\).

The derivative of our equation is \(9x^2+\frac{1}{2\sqrt x}\)

Given our bound \((0.25, 1)\), this is impossible since \(9x^2\geq0\) and \(x>0.25\) so also \(\frac{1}{2\sqrt x}>0\).

Thus, we have a contradiction and our function \(f\) has exactly one root.

\end{enumerate}

\end{enumerate}
\end{document}