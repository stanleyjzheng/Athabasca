\documentclass[11pt, letterpaper, twoside]{article}
\usepackage[letterpaper, portrait, left=1in, right=1in, top=1in, bottom=1in]{geometry}\usepackage{amsmath}
\usepackage{amssymb}
\usepackage{graphicx}
\usepackage[explicit]{titlesec}
\usepackage{epstopdf}
\usepackage{amsmath}
\usepackage{inputenc}
\usepackage{enumitem}
\usepackage{booktabs, multirow} %for borders and merged ranges
\usepackage{soul}% for underlines
\usepackage[table]{xcolor} % for cell colors
\newcommand\aug{\fboxsep=-\fboxrule\!\!\!\fbox{\strut}\!\!\!} %Use \aug to make a equal column for an augmented matrix. Ie. 1 & 2 & 3 & 4 & \aug & x \\
\begin{document}
\begin{titlepage}
\centering
\vspace*{60px}
\hspace{0pt}
\includegraphics[width=0.2\textwidth]{logo}\par\vspace{1cm}
{\scshape\LARGE Athabasca University \par}
\vspace{1cm}
{\scshape\Large MATH 265\par}
\vspace{1.5cm}
{\huge\bfseries Assignment 4\par}
\vspace{2cm}
{\Large\itshape Stanley Zheng\par}
\vfill
{\large September 3, 2020\par}
\vspace*{50px}
\hspace{0pt}
\pagebreak
\end{titlepage}
\begin{enumerate}
\item \begin{enumerate}[label=\alph*)]
\item The domain of the function \(g\) is \((-\infty, x)\cup(x, y)\cup(y, z]\)
\item 
\end{enumerate}
\end{enumerate}
\end{document}
