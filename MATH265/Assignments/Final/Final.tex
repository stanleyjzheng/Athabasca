\documentclass[11pt, letterpaper, twoside]{article}
\usepackage[letterpaper, portrait, left=1in, right=1in, top=1in, bottom=1in]{geometry}\usepackage{amsmath}
\usepackage{amssymb}
\usepackage{graphicx}
\usepackage[explicit]{titlesec}
\usepackage{epstopdf}
\usepackage{amsmath}
\usepackage{inputenc}
\usepackage{enumitem}
\usepackage{booktabs, multirow} %for borders and merged ranges
\usepackage{soul}% for underlines
\usepackage[table]{xcolor} % for cell colors
\newcommand\aug{\fboxsep=-\fboxrule\!\!\!\fbox{\strut}\!\!\!} %Use \aug to make a equal column for an augmented matrix. Ie. 1 & 2 & 3 & 4 & \aug & x \\
\begin{document}
\begin{titlepage}
\centering
\vspace*{60px}
\hspace{0pt}
\includegraphics[width=0.2\textwidth]{logo}\par\vspace{1cm}
{\scshape\LARGE Athabasca University \par}
\vspace{1cm}
{\scshape\Large MATH 265\par}
\vspace{1.5cm}
{\huge\bfseries Final\par}
\vspace{2cm}
{\Large\itshape Stanley Zheng \par}
\vfill
{\large October 23, 2020\par}
\vspace*{50px}
\hspace{0pt}
\pagebreak
\end{titlepage}
\begin{enumerate}
\item %Q1
First, we can differentiate with respect to \(x\) as per the fundamental theorem of Calculus.
The fundamental theorem of Calculus states that if \(g(x)=\int_a^xf(t)dt\), then \(g^\prime(x)=f(x)\).

We will use the property \(\int_a^bf(x)dx=F(b)-F(a)\).

In this case, we have \(\frac{d}{dx}F(x)=\frac{d}{dx}\left(\int^x_af(t)dt\right)\). 
We also know that \(F(x)-F(a)=\int_x^a\frac{f(t)}{t^2}dt\)

\[\frac{d}{dx}(5+\int_a^x\frac{f(t)}{t^2})=12\frac{d}{dx}\sqrt{x}\]
\[\frac{d}{dx}(F(x)-F(a))=\frac{6}{\sqrt{x}}\]

Since \(F(x)\) is the antiderivative of \(f(x)\), we have \(\frac{d}{dx}F(x)=\frac{f(x)}{x^2}\).
In addition, \(F(a)\) is 0 since the derivative of a constant is 0.

\[\frac{f(x)}{x^2}=\frac{6}{\sqrt{x}}\]
\[f(x)=6x^{\frac{3}{2}}\]

Substituting back in, we have 
\[5+\int_a^x\frac{6t^{\frac{3}{2}}}{t^2}dt=12\sqrt x\]

If we let \(x=a\), the integral becomes 0.
\[5=12\sqrt a\]
\[\sqrt a=\frac{5}{12}\]
\[a=\frac{25}{144}\]
As such, our function is \(f(x)=6x^{\frac{3}{2}}\) and our value \(a\) is $\frac{25}{144}$.

\item %Q2
We can first represent the cost of each panel. 
Since the box has a square base, both the top and bottom areas can be represented by \(x\), and the height can be represented by \(y\).

The cost of the bottom panel can be represented by the surface area times the price per square foot, so the price is \(0.35x^2\).

The top of the box is also square with edge length \(x\), so the price is \(0.15x^2\).

The price for the sides of the box is \(0.20xy\), and since there are 4 side panels, we multiply by 4.

The equation for our total cost is \(0.35x^2+0.15x^2+4(0.20xy)=1\).
The equation for our volume is \(x^2y\), which we must maximize.

Solving for \(y\) in our total cost, we have \(0.5x^2+0.8xy=1\), or \(y=\frac{-5x^2+10}{8x}\).
Substituting into our volume equation, we get 
\begin{align*}
&x^2\left(\frac{5(-x^2+2)}{8x}\right)\\
&=\frac{-5x^3+10x}{8}
\end{align*}

We can now maximize \(f(x)\), which represents the volume.
\begin{align*}
f^\prime(x)&=\frac{8(-15x^2+10)-0(-6x^3+10x)}{8^2}\\
&=\frac{-15x^2+10}{8}
\end{align*}
Now letting \(f^\prime(x)=0\), we can solve find the critical points of \(f(x)\).
\begin{align*}
0&=\frac{-15x^2+10}{8}\\
15x^2&=10\\
x&=\pm \sqrt{\frac{2}{3}}
\end{align*}
We can discard the negative square root, since dimensions can only be positive. Then, we have \(x\approx0.82\).
We can find the \(y\) value by substituting back into our cost equation 
\[0.5(0.82)^2+0.8(0.82)y=1\]
\[y\approx1.02\]
Finally, we can find the volume
\[0.82^2\cdot1.02\approx0.68\]

Therefore, the width of the base and top is approximately $0.82 ft$, the height of the box is $1.02 ft$, and the volume is $0.68ft^3$.
\item \begin{align*}
\int \frac{x+2}{\sqrt[3]{x^2}}dx&=\int \left(\sqrt[3]{x}+\frac{2}{\sqrt[3]{x^2}}\right)dx\\
&=\int \sqrt[3]{x}dx+2\int\frac{1}{\sqrt[3]{x^2}}dx\\
&= \frac{3\sqrt[3]{x^4}}{4}+C+2\int\frac{1}{\sqrt[3]{x^2}}dx\\
&= \frac{3\sqrt[3]{x^4}}{4}+2(3\sqrt[3]x)+C\\
&= \boxed{\frac{3\sqrt[3]{x^4}}{4}+6\sqrt[3]x+C}
\end{align*}

\item %Q4
First, we can rewrite this integral.
\[\int x^2\sqrt{2-x}\]

To solve this integral, we must use \(u\) substitution. 
Let \(u=2-x\), so \(dx=-du\) and \(x^2=(2-u)^2\), then substitute in.
\begin{align*}
-\int(u-2)^2\sqrt u du &= -\int u^{\frac{5}{2}}-4u^{\frac{3}{2}}+4u^{\frac{1}{2}}du\\
&=-\frac{2u^{\frac{7}{2}}}{7}+4\int u^{\frac{3}{2}}du-4\int u^{\frac{1}{2}}du \\
&=-\frac{2u^{\frac{7}{2}}}{7}+4\left(\frac{2u^{\frac{5}{2}}}{5}\right)-4\int u^{\frac{1}{2}}du\\
&=-\frac{2u^{\frac{7}{2}}}{7}+4\left(\frac{2u^{\frac{5}{2}}}{5}\right)-4\left(\frac{2u^{\frac{3}{2}}}{3}\right)\\
&=-\frac{2u^{\frac{7}{2}}}{7}+\frac{8u^{\frac{5}{2}}}{5}-\frac{8u^{\frac{3}{2}}}{3}
\end{align*}
Finally, we substitute \(2-x=u\).
\[\boxed{-\frac{2(2-x)^{\frac{7}{2}}}{7}+\frac{8(2-x)^{\frac{5}{2}}}{5}-\frac{8(2-x)^{\frac{3}{2}}}{3}+C}\]

\item %Q5
First, we can find any vertical asymptotes by letting the denominator equal zero and solving.
However, we can see that \(\sqrt{16x^6+1}=0\) has no real solutions since \(16x^6\) cannot be negative.

As a result, we know that there are no vertical asymptotes. 
To find horizontal asymptotes, we can simply evaluate the limit \(f(x)\) as it approaches infinity and negative infinity.
\[\lim_{x\to\infty}\frac{7x^3}{\sqrt{16x^6+1}}\]
Then, divide the numerator and denominator by \(|x^3|\)
\begin{align*}
\lim_{x\to\infty}\frac{7x^3}{\sqrt{16x^6+1}}&=\lim_{x\to\infty}\frac{\frac{7x^3}{|x^3|}}{\frac{\sqrt{16x^6+1}}{|x^3|}}\\
&=\lim_{x\to\infty}\frac{7}{\sqrt{\frac{16x^6}{x^6}+\frac{1}{x^6}}}\\
&=\frac{7}{4}
\end{align*}
Similarly, we can find the limit of \(f(x)\) as \(x\) approaches negative infinity.
However, since we are considering only negative values of \(x\), then \(\sqrt{x^2}=|x|=-x\) for \(x<0\).
Using this, we have 
\begin{align*}
\lim_{x\to-\infty}\frac{7x^3}{\sqrt{16x^6+1}}&=\lim_{x\to\infty}\frac{\frac{7x^3}{|x^3|}}{\frac{\sqrt{16x^6+1}}{|x^3|}}\\
&=\lim_{x\to-\infty}\frac{\frac{7x^3}{-x^3}}{\frac{\sqrt{16x^6+1}}{\sqrt{x^6}}}\\
&=\lim_{x\to-\infty}\frac{-7}{\sqrt{16+\frac{1}{x^6}}}\\
&=-\frac{7}{4}
\end{align*}
As such, \(f(x)\) has no vertical asymptotes, but two horizontal asymptotes at \(\boxed{\pm\frac{7}{4}}\)

\item %Q6
We can represent the displacement of the particle with a definite integral.
\[\int_0^58t-8\]
To solve this, we can first solve the indefinite integral.
\[8\int tdt-8\int1dt\]
\[=4t^2-8t+C\]
However, we cannot simply substitute in \(t=5\) and subtract from \(t=0\). 
As the question specifies distance, we cannot use the absolute value definite integral, as this is displacement.

We need to find the point \(z\) where the line crosses the \(t\) axis, then calculate \(-\int_0^zv(t)+\int_z^5v(t)\)
To do this, we set \(v(t)=0\).
\[8t-8=0\]
\[t=1\]

We can then find \(\int_0^1v(t)\) by substituting \(x=1\) into our indefinite integral and subtracting from \(x=0.\)
\[\int_0^1v(t)=(4(1^2)-8(1)+C)-(4(0^2)-8(0)+C)=-4\]
Then, find \(\int_1^5v(t)\) in the same manner.
\[\int_1^5v(t)=(4(5^2)-8(5)+C)-(4(1^2)-8(1)+C)=64\]

Finally, we can substitute our integrals into \(-\int_0^zv(t)+\int_z^5v(t)\).
\[\int_0^zv(t)-\int_z^5v(t)=-(-4)+64=68\]

As a result, the total distance travelled of the particle is \(\boxed{68m}\)
\end{enumerate}
\end{document}